\documentclass{beamer}
 
\title{Generalized Search Trees}
\author{Mihail Bogojeski, Alexander Svozil}
\date{\today}
 
\begin{document}
\maketitle
\begin{frame}
\frametitle {Index}
\tableofcontents

\end{frame}

 
\section {What is GiST?}
\begin{frame}%%Eine Folie
  \frametitle{GiST Definition} %%Folientitel
  \textbf {What is Gist?}
  \begin {itemize}
  \item Data Structure that can be used to build a variety of height-balanced search trees.
  \pause \item Makes no assumptions about type of data being stored or queries being serviced
  \pause \item Allows easy implementations of well known indexed trees like B+-Trees, R-Trees
  \end {itemize}
\end{frame}

\section {How is GiST different?}
\begin{frame}%%Eine Folie
  \frametitle{How is GiST different?} %%Folientitel
  
  \begin {itemize}
  \item  B+-Tree Functions 
  \begin {itemize}
    \item range predicates (e.g. $c_1 \leq i \leq c_2$)
  \end {itemize}
  \pause \item R-Tree Functions 
  \begin {itemize}
    \item region predicates (e.g. "find all \emph{i} such that  ($x_1,y_1,x_2,y_2$) overlaps \emph{i} ")
  \end {itemize} 
  \pause \item GiST Functions
  \begin {itemize}
    \item GiST can work with any arbitrary predicate and data type (with any number of free variables)
  \end {itemize} 
  
  \end {itemize}
\end{frame}

\section {Why GiST?}
\begin{frame}%%Eine Folie
  \frametitle{Why use GiST?} %%Folientitel
  
  \begin {itemize}
  \item Extensibility in the conext of database systems
  \pause \item Allows the easy evolution of a database system to support new tree-based indexes
  \pause \item Allows developers to focus on new features of index types without becoming experts in database system internals
  \end {itemize}
\end{frame}

\section {Implementation of GiST}
\begin{frame}%%Eine Folie
  \frametitle{GiST implementation} %%Folientitel
  \begin {itemize}
  \item Structure is similar to normal balanced trees
  \pause \item Each node (except root) contains kM to M entries
  \pause \item Internal node entries: (predicate, pointer to child node)
  \pause \item Leaf node entries: (predicate, pointer to actual data)
  \end {itemize}
\end{frame}
\begin{frame}%%Eine Folie
  \frametitle{Tree Functions} %%Folientitel
  \begin {itemize}
  \item $search :: Predicate \rightarrow GiST \rightarrow [LeafEntry]$
  \pause \item $insert :: Entry \rightarrow GiST \rightarrow Level \rightarrow GiST$
  \pause \item $chooseSubtree :: GiST \rightarrow GiST \rightarrow Entry \rightarrow GiST $
  \pause \item $split ::  GiST \rightarrow Node \rightarrow Entry \rightarrow GiST$
  \pause \item $adjustKeys :: GiST \rightarrow Node \rightarrow GiST$
  \pause \item $delete :: LeafEntry \rightarrow GiST \rightarrow GiST$
  \pause \item $condenseTree ::  GiST \rightarrow Node \rightarrow GiST$
  \end {itemize}
\end{frame}
\begin{frame}%%Eine Folie
  \frametitle{Key Functions} %%Folientitel
  \begin {itemize}
  \item $consistent :: Entry \rightarrow Predicate \rightarrow Bool$
  \pause \item $union ::  [Entry] \rightarrow Predicate$
  \pause \item $penalty :: Entry \rightarrow Entry \rightarrow Integer$
  \pause \item $pickSplit ::  [Entry] \rightarrow [[Entry]]$
  \end {itemize}
\end{frame}
\begin{frame}
  \frametitle{Summary}
  \begin {itemize}
  \item What is GiST?
  \item Why use GiST?
  \item Implementation of GiST
  \item Tree and Key Functuions
  \end {itemize}
\end{frame}

\end{document}
